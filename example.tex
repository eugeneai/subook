\documentclass[14pt, openany, twoside, final]{extbook} % Computer Modern font calls
%\usepackage[usenames]{color}
%\usepackage{fancybox}
%\usepackage{algorithmic} % noend
%\usepackage[boxed]{algorithm} % boxed, ruled, plain (also)
%\usepackage{longtable}%длинные таблицы

\usepackage[final]{graphicx}
\usepackage{algorithm}

% Main style definition
% ---------------------
% ISU standard handbook
%\usepackage[fontdir="",handbook,fancybot,times,inconsolata,smalltitles,microtyping]{subook}
\usepackage[handbook,times,fancybot,inconsolata,smalltitles,microtyping,demo]{subook}
% remove demo in a real project.

% ISU standard monograph (less restrictive and more artistic)
% \usepackage[monograph,mag,times,smalltitles,fancybot,listbib,ptfonts,microtyping]{subook}

% Some artistism for monograph
% ----------------------------
%\makeatletter{}
%\renewcommand\su@chapter@font{\sffamily\sfcpshape\bfseries}
%\renewcommand\su@chapter@font@size{\LARGE}
%\makeatother{}
%\floatname{algorithm}{Процедура}
%\renewcommand{\listalgorithmname}{Список процедур}
%\renewcommand\cftsecnumwidth{5ex}
%\tolerance=5000
%\renewcommand{\chaptername}{Глава}



\usepackage{tikz}
%\usetikzlibrary{arrows,arrows.meta,shapes}

\long\def\rem#1{}
%\def\AR{{\em Прим.~автора~пособия}}
\def\emphbib#1{#1}
\newenvironment{questions}{\subsubsection*{Вопросы для самопроверки}\begin{enumerate}\itemsep0pt minus 0.3pt\parskip0pt plus 0.3pt}{\end{enumerate}}

\newtheorem{example}{Пример}[chapter]
\hypersetup{
    bookmarks=true,         % show bookmarks bar?
    unicode=true,           % non-Latin characters in Acrobat’s bookmarks
    pdftoolbar=true,        % show Acrobat’s toolbar?
    pdfmenubar=true,        % show Acrobat’s menu?
    pdffitwindow=false,     % window fit to page when opened
    pdfstartview={FitH},    % fits the width of the page to the window
    pdftitle={Рекурсивно-Логическое Программирование},    % title
    pdfauthor={Евгений Александрович Черкашин},     % author
    pdfsubject={Методическое пособие},   % subject of the document
    pdfcreator={LaTeX},   % creator of the document
    pdfproducer={LaTeX}, % producer of the document
    pdfkeywords={Пролог} {Искусственный интеллект} {Планирование действий}, % list of keywords
    pdfnewwindow=true,      % links in new window
    colorlinks=true,       % false: boxed links; true: colored links
    linkcolor=[rgb]{0 0.4 0.1},          % color of internal links (black)
    citecolor=blue,        % color of links to bibliography
    filecolor=black,      % color of file links
    urlcolor=[rgb]{0.3 0.0 0.3}           % color of external links
}

%\renewcommand{\headrulewidth}{1pt}

\clubpenalty=3000
\widowpenalty=3000
%\brokenpenalty=10000
%\floatingpenalty=10000

%% \setdefaultlanguage{russian}
%% \setmainlanguage{russian}
%% \setotherlanguage{english}

\renewcommand\baselinestretch{1.1}
\parskip=0pt plus 0.3pt
\begin{document}
% \itemsep3pt plus 0pt minus 3pt
% \widowpenalty=10000
% \clubpenalty=10000
% \renewcommand\sutitlefontface{\Large\ptsans\nwshape\bfseries}
% \theorembodyfont{\rmfamily}

\lstset{language=Prolog, morecomment=[l]{\%}}
\renewcommand{\chaptername}{} % for ISU Handbooks
\renewcommand{\refname}{Рекомендуемая литература} % ... also
\renewcommand{\bibname}{\refname}
\begin{titlepage}
\thispagestyle{empty}
\begin{center}{\small{}
Министерство образования и науки
Российской федерации \\
Федеральное государственное бюджетное образовательное\\
учреждение высшего профессионального образования\\
<<Иркутский государственный университет>>\\[2ex]
    Учреждение Российской академии наук \\
<<Институт динамики систем и теории управления \\
Сибирского отделения РАН>>
}
\vfill
\hbox to \linewidth{\hfill\bfseries Е.~А.~Черкашин\hfill}
 \vspace{2em}
{\large\bfseries Рекурсивно"=логическое программирование}\\
 \vspace{2em}
{Учебное пособие}
\vfill
%\vfill
\vfill
 \textbf{Иркутск 2013}
\end{center}
\end{titlepage}
%\newpage
%\begingroup
%\normalfont \ttfamily Это \textbf{текст} в \itshape нормальном \bfseries фонте\sffamily етноф\nwshape фонт\rmcpshape тноф\sfcpshape фонт\ttfamily тноф...
%\endgroup

\newpage
\begin{mygroup}
\thispagestyle{empty}
\noindent УДК 681.3.06 (075.8)\\ % (075.8) = Handbook
% \noindent УДК~165:681.3.06+62-52\\

% Программирование для ЭВМ--Учебники и пособия для вузов
\noindent ББК 32.973-01я73\\ % я73 - Учебник для вузов -01=

% 86.4 = Логика
% 32.973 = Программное обеспечение
% 32.813 = Искусственный интеллект
% -01 =
% ББК 87.4:32.973-01+32.813\\ % monograph
\noindent\mbox{}\hspace{2em}Ч-48 % Author Sign (Авторский знак)

\begin{center}\small
Печатается по решению ученого совета ИМЭИ\\[2ex]
\bfseries Издание выходит в рамках Программы\\
стратегического развития ФГБОУ ВПО <<ИГУ>>\\
на 2012--2016 гг., проект Р121-02-001
\end{center}
\vspace{1ex}
\begin{center}\small
\textbf{Рецензенты:} \\
канд.~техн.~наук~{\em В.~С.~Ульянов},\\ канд.~физ.-мат.~наук~{\em А.~А.~Лемперт}
\end{center}
\vfill
\noindent\begin{minipage}[t]{2em}
\noindent\mbox{}\\
Ч-48
\end{minipage}%
\begin{minipage}[t]{0.95\linewidth}
\setlength{\parindent}{5ex}
\noindent{\bfseries Черкашин~Е.~А.}

Рекурсивно"=логическое программирование\,{}: учеб.~пособие\,/~Е.~А.~Черкашин.~--
Иркутск\,: Изд-во ИГУ, 2013.~-- \pageref{lastpage}~c.

{\bfseries ISBN 978-5-9624-0938-2}
\vspace{2ex}

\begingroup\small\parskip0pt
\vspace{1ex}
В пособии представлены лекционные материалы и лабораторные работы курса <<Рекурсивно"=логическое программирование>>: базовые термины искусственного интеллекта, задачи, методы и их свойства; основы рекурсивно"=логического программирования на языке Пролог; типичные задачи, решение которых лаконично представляется как рекурсивные и переборные алгоритмы. Пособие содержит задания на лабораторный практикум по темам <<Формализация>>, <<Обработка списков>>, <<Метод Британского музея (отобразить и проверить)>> и <<Базы данных>>.

    Пособие предназначено для студентов специальности <<инженер"=программист>>, <<инженер\,---\,системный программист>>. Изучение материала будет полезно студентами других специальностей, так или иначе связанных с программированием, формальной логикой и комбинаторикой.

\mbox{}
\endgroup
\end{minipage}
\mbox{}\hspace{0.7\linewidth}
\begin{minipage}{0.3\linewidth}\small
\noindent УДК 681.3.06 (075.8)\\
\noindent ББК 32.973-01я73
\end{minipage}

\vfill
\noindent\begin{minipage}[t]{0.35\linewidth}\small
\noindent ISBN 978-5-9624-0938-2
\end{minipage}%
\begin{minipage}[t]{0.65\linewidth}\small
\begin{itemize}
\setlength{\itemsep}{-0.5ex}
\setlength{\parsep}{0pt}
\item[\copyright{}] Черкашин~Е.~А., 2014
\item[\copyright{}] ФГБОУ ВПО <<ИГУ>>, 2014
%\item[\copyright{}] ФГБОУ ВПО НИ ИрГТУ, 2014 % TODO: Check the name
\item[\copyright{}] Институт динамики систем и теории управления СО РАН, 2014
\end{itemize}
\end{minipage}
\end{mygroup}
\clearpage
%\setcounter{page}{2}
\tableofcontents
\clearpage

\newpage
\chapter*{Предисловие}



Автор является приверженцем открытых технологий, свободных книг, научного метода познания мира и открытого программного обеспечения. Адрес исходного кода методического пособия~--- \url{https://github.com/eugeneai/ais/tree/new-isu}. Исхоный код разрешено использовать в соответствии с лицензией  \foreignlanguage{english}{CC BY-NC-SA 4.0 (Attribution-NonCommercial-ShareAlike 4.0 International)}, которая разрешает использование материала в своих произведениях (необходимо указывать автора оригинальных материалов), запрещает коммерческое использование материалов (ввиду наличия вышеперечисленных заимствований) и требует распространение производных материалов производить по этой же самой лицензии (по предыдущей причине). Адрес лицензии~---  \url{http://creativecommons.org/licenses/by-nc-sa/4.0/}.

В тексте пособия использована следующая разметка:
\begin{description}
\item[Жирным шрифтом] выделяются имена существительные и глаголы, на которые, по мнению автора, следует обратить внимание, это~--- что-то вроде дополнительной семантической разметки текста учебного пособия.
\item[\normalfont{\tt Моноширинным шрифтом}] приводятся программы, отрывки программ в основном тексте пособия, а также имена идентификаторов, т.~е. все, что имеет какое"=либо отношение к {\bf тексту программы}.
\item[\normalfont{\em Наклонным шрифтом}] выделяются {\bf новые} термины, вводимые в текст и возникающие, например, в определениях, а также текст выделенных примеров.
\item[\normalfont При помощи <<кавычек>>] выделяются метафоры, значения, элементы текстов программ, цитаты, слова, использованные в переносном смысле, и т.~д.
%\item[\normalfont{\sf Рубленым шрифтом}] декорируются тексты, которые надо как-то особо выделить на общем фоне.
\end{description}

\medskip
\noindent\hbox to \linewidth{\hfill\sf Старший~научный сотрудник ИДСТУ СО РАН,}
\noindent\hbox to \linewidth{\hfill\sf доцент кафедры ИТ ИМЭИ ИГУ}
\noindent\hbox to \linewidth{\hfill\sf кандидат~технических~наук}
\noindent\hbox to \linewidth{\hfill\sf Е.~А.~Черкашин}
%\noindent\hbox to \linewidth{\hfill\sf аспирант кафедры ИТ ИМЭИ ИГУ}
%\noindent\hbox to \linewidth{\hfill\sf И.~Н.~Терехин}

\vfill
\makeatletter
\noindent{\sf P.~S.} Автора всегда можно найти по адресу \href{mailto:eugeneai@icc.ru}{\tt{}eugeneai@icc.ru}, в поле <<{\tt тема}>> прошу указывать <<TODO-2013>>.
\makeatother

\chapter{Раздел}


\section{Терминология...}


\subsection{И т.п.}


\begin{questions}
\item{} Перечислите .....
\item{} Дайте характеристику ...
\item{} В чем суть ....
\item{} Приведите ...
\item{} Какие ...
\end{questions}

\chapter{Другой раздел}


\chapter*{Заключение}


Изучение ...

В книге ...


%\listoffigures
%\addcontentsline{toc}{section}{Список иллюстраций}
%\listoftables
%\addcontentsline{toc}{section}{Список таблиц}
\begin{thebibliography}{99}\itemsep1pt \parskip 0pt plus 0.3pt
\bibitem{Anderson} Андерсон~Р. \emphbib{Доказательство правильности программ}\,{}: пер. с англ.\,{}/ Р.~Андерсон. -- М.\,:\,Мир, 1982. -- 168~c.: ил.
\bibitem{Bratko} Братко~И. \emphbib{\href{http://royallib.ru/book/bratko_ivan/programmirovanie_na_yazike_prolog_dlya_iskusstvennogo_intellekta.html}{Программирование на языке ПРОЛОГ для искусственного интеллекта}}\,{}: пер. с англ.\,/ И.~Братко. -- М.\,:~Мир, 1990. -- 560~c.: ил.
\bibitem{Vass:2000} Васильев~С.~Н. \emphbib{\href{http://bookfi.org/book/616050}{Интеллектное управление динамическими системами}}\,{}/ С.~Н.~Васильев, А.~К.~Жерлов, Е.~А.~Федосов, Б.~Е.~Федунов. -- М.\,:~Физматлит, 2000. -- 352~с: ил.
\bibitem {AIDictionary} \emphbib{\href{http://aihandbook.intsys.org.ru/index.php/intro/ai-handbook}{Искусственный интеллект\,{}: в 3~кн.}}\,{}/ под ред. Э.~В. Попова. -- М.\,:~Радио и связь, 1990. -- 464 c.:\,{}ил.
\bibitem{Lauriere} Лорьер.~Ж.-Л.  \emphbib{\href{http://publ.lib.ru/ARCHIVES/L/LOR'ER_Jan_Lui/_Lor'er_J.L..html}{Системы искусственного интеллекта}\,{}: пер. с франц.}\,{}/ Ж.-Л. Лорьер. -- М.\,:~Мир, 1991. -- 568~с.: ил.
\bibitem{Malpas} Малпас~Дж. \emphbib{\href{http://padaread.com/?book=40731&pg=1}{Реляционный язык Пролог и его применение}}\,{}/ Дж.~Малпас. -- М.\,:~Наука, 1990. -- 464~с.
\bibitem{math_slov:88} \emphbib{\href{https://app.box.com/shared/793ukgvblxmj0hh6btw4}{Математический энциклопедический словарь}}\,{}/ гл.~ред. Ю.~В.~Прохоров. -- М.\,:~Сов.~энциклопедия, 1988. -- 847~c.
\bibitem{DDW} Непейвода~Н.~Н. \emphbib{\href{http://www.logic-books.info/taxonomy/term/215}{Прикладная логика\,{}: учеб. пособие}}\,{}/ Н.~Н.~Непейвода. -- 2-е изд. -- Новосибирск\,{}:~Изд-во Новосиб. ун-та, 2000. -- 521~c.: ил.
\bibitem{DDWII} Непейвода~Н.~Н.  \emphbib{\href{http://philosophy.ru/library/logic_math/library/nepeivoda_prog.pdf}{Основания программирования}}\,{}/ Н.~Н.~Непейвода, И.~Н.~Скопин. -- Москва; Ижевск\,{}:~Институт компьютерных исследований, 2003 -- 880~c.: ил.
\bibitem {Russell} Рассел~С. \href{http://www.aiportal.ru/downloads/books/ai-modern-approach-2-edition-by-rassel-norvig.html}{Искусственный интеллект: современный подход}\,{}: пер. с англ.\,{}/ С.~Рассел, П.~Новриг. 2-е изд. -- М.\,:~Изд. дом <<Вильямс>>, 2006. -- 1408~c.: ил.
\bibitem{WIKI-DCG} \emphbib{\href{https://en.wikipedia.org/wiki/Definite_clause_grammar}{DC-грамматика}} [Электронный ресурс]\,{}// Wikipedia, The Free Encyclopedia\,{}: сайт. -- URL:\texttt{https://en.wikipedia.org/wiki/Definite\_clause\linebreak\_grammar}. (дата обращения: 28.11.2013).
\bibitem{GNUP} \emphbib{\href{http://www.gprolog.org/}{The GNU Prolog web site [Электронный ресурс]\,{}: сайт}}. URL:\url{http://www.gprolog.org/}. (дата обращения: 28.11.2013).
\bibitem{SWIP} \emphbib{\href{http://www.swi-prolog.org/}{SWI-Prolog's home [Электронный ресурс]\,{}: сайт}}. URL:\url{http://www.swi-prolog.org/}. (дата обращения: 28.11.2013).
\end{thebibliography}
\label{lastpage}
\newpage
\thispagestyle{empty}
\mbox{}

\vfill\vfill\vfill\vfill

\hfill{}{\small\itshape Учебное издание}
\vspace{4ex}
\begin{center}
{\small\textbf{Черкашин} Евгений Александрович\\[1em]}
{\bfseries Рекурсивно"=логическое программирование}\\[1em]
ISBN~978-5-9624-0938-2
\vfill

\small
Редактор \textit{Г.~А.~Борисова}\\
Верстка \textit{Е.~А.~Черкашин}

\vfill{}
{\small Макет подготовлен при помощи системы \LuaLaTeX\\\mbox{}}
\vfill{}

Темплан 2013\,{}г. Поз.\,{}186

\end{center}
\begin{center}\small
\noindent Подписано в печать 28.12.2013.
Формат~60$\times$90 1/16.\\  %Гарнитура \sutypeface{}.
%Верстка \LuaLaTeXe.
%Бумага офсетная. Печать офсетная. Усл.печ.л.
Уч.-изд.\,{}л.\,{}6,4. Усл.\,{}печ.\,{}л. 6,8. Тираж~100~экз. Заказ~170
\end{center}
\vspace{1           ex}
\begin{center}\small
Издательство ИГУ\\{}
664003, г.\,{}Иркутск, бульвар Гагарина, 36
\end{center}
\end{document}
%%%%%%%%%%%%%%%%%%%%%%%%%%%%%%%%%%%%%%%%%%%%%%%%%%%
%%% Local Variables:
%%% TeX-engine: lualatex
%%% End:
