\documentclass[a5paper,10pt]{extbook}
\usepackage[english,russian]{babel}

%% \usepackage{polyglossia}
%% \setdefaultlanguage{english}
%% \setotherlanguage{russian}

\usepackage{luacode}
\usepackage{fontspec}
\usepackage{blindtext}
\usepackage{microtype}
%\usepackage[a5]{geometry}
%\setmainfont{Fira Sans OT}
%\setmainfont[Ligatures={TeX}]{Fira Sans OT}
%\setmainfont[Ligatures={TeX}]{Times New Roman}
%\setmainfont[Ligatures={TeX}]{Linux Libertine}
%\setmainfont[Ligatures={TeX}]{Linux Libertine O}
%\setmainfont[Ligatures={TeX}]{Georgia}    % !!!!
%\setmainfont[Ligatures={TeX}]{Constantia}
\setmainfont[
BoldFont=quantantiquabold.ttf,
ItalicFont=quantantiquaitalic.ttf,
BoldItalicFont=quantantiquabolditalic.ttf,
]{quantantiquaplain.ttf}


\setmonofont{Fira Mono OT}

\begin{document}

\selectlanguage{english}
%\Blinddocument{}

\selectlanguage{russian}
\chapter{Глава тестовая}
Случайное число (Number):
\begingroup
\tt\bfseries
\begin{luacode}
tex.print(math.random())
\end{luacode}

ffi
\endgroup

\long\def\fonttest#1{
\begingroup
\fontspec{#1}
\newpage{}
Шрифт: \texttt{#1} Букв тест К~~А~~С~~~далее.\par

Действие романа происходит через 500 лет, после того как была создана Академия [1]. Считалось, что Вторая Академия (которая пытается управлять Первым Основанием, используя науку"=психоисторию) была уничтожена [2]. Однако нашлись сомневающиеся в этом факте. Член Совета Голан Тревиз, в прошлом — офицер космофлота, считает что Вторая Академия всё ещё существует и тайно управляет событиями. По приказу мэра на Терминусе, столицы Федерации Академии, его арестовывают и обвиняют в государственной измене и высылают с Терминуса с приказом найти Вторую Академию. С ним вместе отправляется Янов Пилорат, профессор древней истории и мифолог, заинтересованный найти местоположение Земли, мифического родного мира человечества. В то же время Стор Джиндибел, молодой и энергичный Спикер Второй Академии, пытается разыскать некую третью силу, которая тайно управляет событиями в галактике, включая действия Второй Академии.

Роман был впервые опубликован в сентябре 1982 года издательством Doubleday. Он был написан почти через тридцать лет после изначальной трилогии цикла, благодаря давлению поклонников и издателей, а также внушительному гонорару. С тех пор роман неоднократно переиздавался на английском языке, а также был переведён на несколько других языков, включая русский [3].
Роман номинировался на ряд престижных премий, включая «Небьюлу», а также в 1983 году был удостоен премий «Хьюго» и «Локус» [3].

\textbf{Русскоязычные издания}
На русском языке роман был впервые опубликован в 1992 году под названием «Край Основания» в издательстве «Орис» [4]. Переводчик романа указан не был[5]. В 1993 году роман был издан в другом переводе издательством «Орел» под названием «Предел Фонда» [6]. Переводчик также не был указан [7].
В 1994 году роман был издан под названием «Край Академии» в серии «Миры Айзека Азимова» издательством «Полярис»[8]. Перевод для этого издания выполнила Надежда Сосновская[9]. В 1997 году роман в этом же переводе был переиздан в серии «Хроники Академии» [10].

В дальнейшем роман несколько раз переиздавался в переводе Н. Сосновской под названием «Академия на краю гибели» издательством Эксмо — в 2000 [11], 2003 [12], 2006 [13], 2007 [14] и 2008 годах [15].

\endgroup
}

\fonttest{Georgia}
\fonttest{Droid Serif}
\fonttest{Free Serif}
\fonttest{Linux Libertine}
\fonttest{Times New Roman}
%\fonttest{Comic Sans MS}
\fonttest{Calibri}
\fonttest{DejaVu Serif}
\fonttest{Proletariat.ttf}
\fonttest{RedBanner.ttf}
\fonttest{Playwright.ttf}
\fonttest{Pilotka.ttf}
\fonttest{LeningradkaKyrsiv.ttf}
\fonttest{academy.ttf}
\fonttest{kudriashov.ttf}
\fonttest{baltica.ttf}




Основной Букв тест К~~А~~С~~~далее.

%\fontspec[
%Path=/usr/share/texmf-dist/fonts/opentype/public/philokalia/
%]{Philokalia-Regular.otf}

More magic here.

Действие романа происходит через 500 лет, после того как была создана Академия [1]. Считалось, что Вторая Академия (которая пытается управлять Первым Основанием, используя науку"=психоисторию) была уничтожена [2]. Однако нашлись сомневающиеся в этом факте. Член Совета Голан Тревиз, в прошлом — офицер космофлота, считает что Вторая Академия всё ещё существует и тайно управляет событиями. По приказу мэра на Терминусе, столицы Федерации Академии, его арестовывают и обвиняют в государственной измене и высылают с Терминуса с приказом найти Вторую Академию. С ним вместе отправляется Янов Пилорат, профессор древней истории и мифолог, заинтересованный найти местоположение Земли, мифического родного мира человечества. В то же время Стор Джиндибел, молодой и энергичный Спикер Второй Академии, пытается разыскать некую третью силу, которая тайно управляет событиями в галактике, включая действия Второй Академии.

Роман был впервые опубликован в сентябре 1982 года издательством Doubleday. Он был написан почти через тридцать лет после изначальной трилогии цикла, благодаря давлению поклонников и издателей, а также внушительному гонорару. С тех пор роман неоднократно переиздавался на английском языке, а также был переведён на несколько других языков, включая русский [3].
Роман номинировался на ряд престижных премий, включая «Небьюлу», а также в 1983 году был удостоен премий «Хьюго» и «Локус» [3].

\textbf{Русскоязычные издания}
На русском языке роман был впервые опубликован в 1992 году под названием «Край Основания» в издательстве «Орис» [4]. Переводчик романа указан не был[5]. В 1993 году роман был издан в другом переводе издательством «Орел» под названием «Предел Фонда» [6]. Переводчик также не был указан [7].
В 1994 году роман был издан под названием «Край Академии» в серии «Миры Айзека Азимова» издательством «Полярис»[8]. Перевод для этого издания выполнила Надежда Сосновская[9]. В 1997 году роман в этом же переводе был переиздан в серии «Хроники Академии» [10].

В дальнейшем \textit{роман несколько \textbf{раз переиздавался}} в переводе Н. \textbf{Сосновской \textit{под} названием} «Академия на краю гибели» издательством Эксмо — в 2000 [11], 2003 [12], 2006 [13], 2007 [14] и 2008 годах [15].

Ligature check: ffi.

\end{document}
